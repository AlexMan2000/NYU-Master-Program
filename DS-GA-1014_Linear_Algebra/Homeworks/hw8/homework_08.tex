\documentclass[11pt,nocut]{article}

\usepackage{style/packages}
\usepackage{style/notations}
\usepackage{fancyhdr}

\pagestyle{fancy}
\renewcommand{\headrulewidth}{1pt}
\fancyhead[R]{DS-GA 1014---Fall 2023}
\fancyhead[L]{HW8 --- SVD and Graph --- due November 12, 2023 at noon}

\setcounter{section}{8}

\begin{document}

\noindent\textbf{Rules:}
% \centerline{\pgfornament[width=13cm]{89}}
{\small
	\begin{itemize}
		\item Unless otherwise stated, all answers must be mathematically justified.
		\item Partial answers will be graded. 
		\item Your submission has to be uploaded to Gradescope. In Gradescope, indicate the page on which each problem is written.
		\item You can work in groups but each student must write his/her/their own solution based on his/her/their own understanding of the problem. Please list on your submission the students you work with for the homework (this will not affect your grade).
		\item Problems with a $(\star)$ are extra credit, they will not (directly) contribute to your score of this homework. However, for every $4$ extra credit questions successfully answered your lowest homework score get replaced by a perfect score.
		\item If you have any questions, feel free to ask them on Ed Discussion (so that everyone can benefit from the answer) or stop at the office hours.
	\end{itemize}
}
% \vspace{-0.4cm}
% \centerline{\pgfornament[width=13cm]{89}}
% \vspace{0.5cm}


\bigskip

\begin{problem}[2 points]
	Let $A \in \R^{n \times m}$. The Singular Values Decomposition (SVD) tells us that there exists two orthogonal matrices $U \in \R^{n \times n}$ and $V \in \R^{m \times m}$ and a matrix $\Sigma \in \R^{n \times m}$ such that $\Sigma_{1,1} \geq \Sigma_{2,2}  \geq \cdots \geq 0$ and $\Sigma_{i,j} = 0$ for $i\neq j$
	$$
	A = U \Sigma V^{\sT}.
	$$
	The columns $u_1, \dots, u_n$ of $U$ (respectively the columns $v_1, \dots, v_m$ of $V$) are called the left (resp.\ right) singular vectors of $A$. The non-negative numbers $\sigma_i \defeq \Sigma_{i,i}$ are the singular values of $A$. Moreover we also know that $r \defeq \rank(A) = \# \{i \, | \, \Sigma_{i,i} \neq 0 \}$.
	\begin{enumerate}[label=\normalfont(\textbf{\alph*})]
		\item Let 
			$\widetilde{U} = \!
			{\small \begin{pmatrix}
					| & & | \\
					u_1 & \!\cdots\! & u_r \\
					| & & | 
			\end{pmatrix} } \! \in \! \R^{n \times r}$ ,
			$\widetilde{V} = \!
			{\small \begin{pmatrix}
					| & & | \\
					v_1 & \!\cdots\! & v_r \\
					| & & | 
			\end{pmatrix} } \! \in \! \R^{m \times r}$ and
			$\widetilde{\Sigma} = \Diag(\sigma_1, \dots, \sigma_r) \in \R^{r \times r}$.
			Show that $A = \widetilde{U}\widetilde{\Sigma}\widetilde{V}^{\sT}$. \\
   Hint: show that the $ij$th entry of $U\Sigma V^T$ and $\tilde{U}\tilde{\Sigma}\tilde{V}^T$ match for all $i,j$.
 		\item Give orthonormal bases of $\Ker(A)$ in terms of the singular vectors of $A$.
	\end{enumerate}
\end{problem}

\vspace{2mm}

\begin{problem}[1 points] 
Given matrix $A\in\R^{n\times m}$ where $n>m$. Assume $A^\top A\in\R^{m\times m}$ is invertible, we define the projection matrix $$P = A(A^\top A)^{-1} A^\top\in\R^{n\times n}.$$
Show that $P$ is symmetric, and its eigenvalues (in descending order) are given as $\lambda_1=\lambda_2=...=\lambda_m=1$, and $\lambda_{m+1}=...=\lambda_n=0$. 


\noindent
Hint: use the SVD of $A$. 
\end{problem}

\vspace{2mm}


\begin{problem}[4 points] 

In the lecture we focused on the graph Laplacian
	$$
	L = D - A,
	$$
where $A$ is the adjacency matrix of the graph, and $D = \Diag(\deg(1), \dots, \deg(n))$. 
Observe that the eigenvalues of the matrices $A,L$ can scale with the number of vertices in the graph (which may be very large in practical settings). In this question, we will see that by applying proper normalization, the matrix eigenvalues can be much more well-behaved. 

We introduce the normalized adjacency matrix and the normalized Laplacian as follows,
	$$
	\tilde{A} = D^{-1/2} A D^{-1/2}, \quad \tilde{L} = D^{-1/2} L D^{-1/2} = \Id_n - \tilde{A},
	$$
	where $D^{-1/2} = \Diag(\deg(1)^{-1/2}, \dots, \deg(n)^{-1/2})$. Here we assume $\deg(i)\neq 0$ for all $i\in[n]$. 

\begin{enumerate}[label=\normalfont(\textbf{\alph*})]
  \item Show that the smallest eigenvalue of the normalized Laplacian $\lambda_{\min}(\tilde{L}) = 0$. 
  \item Show that the largest eigenvalue of the normalized Laplacian can be written as 
  $$\lambda_{\max}(\tilde{L}) = \max_{v\in\R^n, v\neq 0} \frac{v^\top L v}{v^\top D v}.$$ 
  Based on this formulation, prove that $\lambda_{\max}(\tilde{L}) \le 2$. \\
    \textit{Hint: Use the quadratic form definition $x^\top L x = \sum_{i\sim j}(x_i-x_j)^2$.}   
    \item Show that for all $i\in[n]$, we have $|\lambda_{i}(\tilde{A})|\le 1$, that is, all the eigenvalues of $\tilde{A}$ fall between $-1$ and $1$.  \\
    \textit{Hint: apply the conclusion in the previous question.}   
\end{enumerate}

\end{problem}

\vspace{2mm}


\begin{problem}[3 points]
	The goal of this problem is to use spectral clustering techniques on real data.
	The file \texttt{adjacency.txt} contains the adjacency matrix of a graph taken from a social network. This graphs has $n=328$ nodes (that corresponds to users). An edge between user $i$ and user $j$ means that $i$ and $j$ are ``friends'' in the social network.
	The notebook \texttt{friends.ipynb} contains functions to read the adjacency matrix as well as instructions/questions.
	\\

\noindent
\textbf{Details on coding:} In this question, you should use the \underline{normalized} Laplacian matrix defined in the previous question: 
$$
\tilde{L} = D^{-1/2} L D^{-1/2} = \Id_n - D^{-1/2} A D^{-1/2},
$$
rather than the original $L$. The reason for using this normalized version is that the ``unnormalized Laplacian'' $L$ does not perform well when the degrees in the graph are very broadly distributed, i.e.\ very heterogeneous. In such situations, the normalized Laplacian $\tilde{L}$ is supposed to lead to a more consistent clustering. 
\\

\noindent
\textbf{Submission:} It is intended that you code in Python and use the provided Jupyter Notebook. Please only submit a pdf version of your notebook (right-click $\to$ `print' $\to$ `Save as pdf').
\end{problem}

\vspace{2mm}

\begin{problem}[$\star$]
Let $G$ be a connected graph with $n$ nodes. Define $L \in \R^{n \times n}$ the associate Laplacian matrix and $\lambda_1 \leq \lambda_2 \leq \cdots \leq \lambda_n$ its eigenvalues. Let $G'$ be a graph constructed from $G$ by simply adding an edge. Similarly denote by $\lambda_2'$ its second smallest eigenvalue. Show that  $\lambda_2' \geq \lambda_2$. 
\end{problem}


% \begin{problem}[$\star$]

% A connected graph $G = (V,E)$ is bipartite if and only if $V$ can be partitioned into two \textit{disjoint} sets $V_1,V_2$ such that every edge in $E$ only connects one vertex in $V_1$ with one vertex in $V_2$ (i.e., there is no edge between two vertices in $V_1$, or two vertices in $V_2$).  

% \begin{enumerate}[label=\normalfont(\textbf{\alph*})]
% \item In the previous problem, we proved that the largest eigenvalue of the normalized Laplacian satisfies $\lambda_{\max}(\tilde{L}) \le 2$. 
% Show that if $G$ is bipartite, then equality is achieved: $\lambda_{\max}(\tilde{L}) = 2$. 
% \item Show that if $\lambda_{\max}(\tilde{L}) = 2$, then $G$ must be bipartite.  
% \end{enumerate}

    
% \end{problem}


\end{document}
