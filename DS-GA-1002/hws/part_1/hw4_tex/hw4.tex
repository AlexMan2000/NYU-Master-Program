\documentclass[12pt,twoside]{article}
\usepackage[dvipsnames]{xcolor}
\usepackage{tikz,graphicx,amsmath,amsfonts,amscd,amssymb,bm,cite,epsfig,epsf,url}
\usepackage[hang,flushmargin]{footmisc}
\usepackage[colorlinks=true,urlcolor=blue,citecolor=blue]{hyperref}
\usepackage{amsthm,multirow,wasysym,appendix}
\usepackage{array,subcaption} 
% \usepackage[small,bf]{caption}
\usepackage{bbm}
\usepackage{pgfplots}
\usetikzlibrary{spy}
\usepgfplotslibrary{external}
\usepgfplotslibrary{fillbetween}
\usetikzlibrary{arrows,automata}
\usepackage{thmtools}
\usepackage{blkarray} 
\usepackage{textcomp}
\usepackage[left=0.8in,right=1.0in,top=1.0in,bottom=1.0in]{geometry}

\input{macros}

\newcommand{\ru}{\rnd{ u}  }
\newcommand{\rd}{\rnd{ d}  }
%\newcommand{\rs}{\rnd{ s}  }
\newcommand{\ri}{\rnd{ i}  }
\newcommand{\re}{\rnd{ e}  }
\newcommand{\rQ}{\rnd{ q}  }
\newcommand{\rC}{\rnd{ c}  }


\begin{document}

\begin{center}
{\large{\textbf{Homework 4}} } \vspace{0.2cm}\\
Due October 15 at 11 pm
\\
\end{center}
\input{hwstatement.tex}\\

\begin{enumerate}

\item (Half life)
The half life of a radioactive substance is a way to quantify how rapidly the substance decays. Given a fixed quantity of the substance, the half time is the time that it takes for it to be reduced to half (i.e. half of the radioactive particles have decayed). It is not immediately apparent why the time should be the same for any quantity. Here we show that it is (probabilistically), as long the particles decay following an exponential distribution.  
\begin{enumerate}
\item Let $\rnd{t}$ be a random variable with a pdf of the form
\begin{align}
f_{\rnd{t}}(t) := \begin{cases}
\lambda \exp(- \lambda t), \qquad \text{if $t\geq 0$},\\
0 \qquad \text{otherwise},
\end{cases}
\end{align}
where $\lambda$ is a fixed constant. We define the half life $t_{1/2}$ as the number that satisfies $\P(\rnd{t} > t_{1/2}) = 1/2$. Compute $t_{1/2}$ in terms of $\lambda$. Then explain intuitively why this is a reasonable definition for the half life.
\item Compute $t$ such that $\P( t_{1/2} < \rnd{t} < t) = 1/4$, and express it in terms of only $t_{1/2}$. Explain why the result is consistent with the intuitive meaning of half life.
\item Compute $\P( \rnd{t} > k t_{1/2} )$ for any integer $k$. Again, explain why the result is consistent with the intuitive meaning of half life.
\end{enumerate}

\item (Triangular pdf)
We are interested in fitting a model with a parametric pdf equal to
\begin{align}
f_{w}\brac{x}  = \begin{cases}
 \frac{2x}{w^2}, \qquad & \text{for } 0 \leq x \leq w,\\
0, \qquad & \text{otherwise},
\end{cases}
\end{align}
where the parameter $w$ is nonnegative. 
%\begin{comment}
%Both the pdf and the cdf are plotted in Figure~\ref{fig:triangle}.  
%\begin{figure}[tp]
%% Preamble: \pgfplotsset{width=7cm,compat=1.12}
%% \begin{center}
%\begin{tikzpicture}[scale=0.95]
%\begin{axis}[xmin= -0.1, xmax=2.1, ymin=-0.1, ymax=1.1,xlabel=$x$,
%ylabel=$f_{X} \brac{x }  $, xticklabels={0,w}, xtick={0,2},
%yticklabels={0,$\frac{2}{w}$}, ytick={0,1}]
%\addplot[blue, very thick, domain=0:2, samples=51] { x/2)};
%\addplot[blue, very thick, domain=-0.1:0, samples=2] {0};
%\addplot[blue, very thick, domain=2:2.1, samples=2] {0};
%\addplot[dashed, very thick, samples=2] coordinates {(2,0)(2,1)};
%\end{axis}
%\end{tikzpicture}
%%\end{center}
%\caption{Triangular pdf and the corresponding cdf.}
%\label{fig:triangle}
%\end{figure}
%\end{comment}

\begin{enumerate}
\item The observed values are 1.25, 0.4, 1.5, 1, 1.2. What are the possible values of the parameter $w$? 
\item Compute the likelihood function corresponding to these data and sketch it. 
\item What is the maximum likelihood estimate of $w$? 
\item  Assume that the data are indeed generated by the parametric model with $w:=w_{\op{true}}$. Does the ML estimate systematically underestimate or overestimate the true parameter? 
\item Generate a sample from a random variable with this parametric distribution, where $w:=2$, using a uniform sample from the interval $\sqbr{0,1}$ equal to 0.64.
\end{enumerate}

\item (Planet)
An astrophysicist determines that a good model for the pdf of the temperature in a newly discovered planet is
\begin{align}
f_{\rnd{t}}(t) := \frac{\lambda \exp (-\lambda \abs{t})}{2},
\end{align}
where $t$ can be any real number (in particular it can be negative or positive).
\begin{enumerate}
\item Compute the cdf of $\rnd{t}$.
\item Compute the maximum-likelihood estimate of $\lambda$ from the following data: 5, -50, -1, 100
\item What is the pdf of $\rnd{t}$ conditioned on the event $\rnd{t}>0$?
\end{enumerate}

\item(Temperature)
The tables in \textit{train.csv} and \textit{test.csv} record the daily maximum temperature (TMAX) of Seattle.
\begin{enumerate}
    \item Estimate the pdf of TMAX with the following models on the training set. Compare the pdf with a normalized histogram in the test set. Which model performs better visually?
        \begin{itemize} 
            \item Estimating the parameter of Gaussian distribution with MLE;
            \item Non-parametric KDE with the Gaussian kernel at different bandwidths (e.g. 1, 2, 5).
        \end{itemize}
    \item Repeat the experiment only on \text{July} and \text{August} data. Which model performs better visually? Compare the results with (a) and explain your findings. 
\end{enumerate}

\end{enumerate}
\end{document}
