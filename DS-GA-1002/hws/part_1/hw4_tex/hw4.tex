\documentclass[12pt,twoside]{article}
\usepackage[dvipsnames]{xcolor}
\usepackage{tikz,graphicx,amsmath,amsfonts,amscd,amssymb,bm,cite,epsfig,epsf,url}
\usepackage[hang,flushmargin]{footmisc}
\usepackage[colorlinks=true,urlcolor=blue,citecolor=blue]{hyperref}
\usepackage{amsthm,multirow,wasysym,appendix}
\usepackage{array,subcaption} 
% \usepackage[small,bf]{caption}
\usepackage{bbm}
\usepackage{pgfplots}
\usetikzlibrary{spy}
\usepgfplotslibrary{external}
\usepgfplotslibrary{fillbetween}
\usetikzlibrary{arrows,automata}
\usepackage{thmtools}
\usepackage{blkarray} 
\usepackage{textcomp}
\usepackage[left=0.8in,right=1.0in,top=1.0in,bottom=1.0in]{geometry}

%% Probability operators and functions
%
% \def \P{\mathrm{P}}
\def \P{\mathrm{P}}
\def \E{\mathrm{E}}
\def \Var{\mathrm{Var}}
\let\var\Var
\def \Cov {\mathrm{Cov}} \let\cov\Cov
\def \MSE {\mathrm{MSE}} \let\mse\MSE
\def \sgn {\mathrm{sgn}}
\def \R {\mathbb{R}}
\def \C {\mathbb{C}}
\def \N {\mathbb{N}}
\def \Z {\mathbb{Z}}
\def \cV {\mathcal{V}}
\def \cS {\mathcal{S}}

\newcommand{\RR}{\ensuremath{\mathbb{R}}}

\DeclareMathOperator*{\argmin}{arg\,min}
\DeclareMathOperator*{\argmax}{arg\,max}
\newcommand{\red}[1]{\textcolor{red}{#1}}
\newcommand{\blue}[1]{\textcolor{blue}{#1}}
\newcommand{\green}[1]{\textcolor{ForestGreen}{ #1}}
\newcommand{\fuchsia}[1]{\textcolor{RoyalPurple}{ #1}}

\newcommand{\wrnd}[1]{\widetilde{ #1 } }
\newcommand{\po}{\wrnd{\op{po}}  }

%
%% Probability distributions
%
%\def \Bern    {\mathrm{Bern}}
%\def \Binom   {\mathrm{Binom}}
%\def \Exp     {\mathrm{Exp}}
%\def \Geom    {\mathrm{Geom}}
% \def \Norm    {\mathcal{N}}
%\def \Poisson {\mathrm{Poisson}}
%\def \Unif    {\mathrm {U}}
%
\DeclareMathOperator{\Norm}{\mathcal{N}}

\newcommand{\bdb}[1]{\textcolor{red}{#1}}

\newcommand{\ml}[1]{\mathcal{ #1 } }
\newcommand{\wh}[1]{\widehat{ #1 } }
\newcommand{\wt}[1]{\widetilde{ #1 } }
\newcommand{\conj}[1]{\overline{ #1 } }
\newcommand{\rnd}[1]{\tilde{ #1 } }
\newcommand{\rv}[1]{ \rnd{ #1}  }
\newcommand{\rM}{\rnd{ m}  }
\newcommand{\rx}{\rnd{ x}  }
\newcommand{\ry}{\rnd{ y}  }
\newcommand{\rz}{\rnd{ z}  }
\newcommand{\ra}{\rnd{ a}  }
\newcommand{\rb}{\rnd{ b}  }
\newcommand{\rt}{\rnd{ t}  }
\newcommand{\rs}{\rnd{ s}  }


\newcommand{\rpc}{\widetilde{ pc}  }
\newcommand{\rndvec}[1]{\vec{\rnd{#1}}}

\def \cnd {\, | \,}
\def \Id { I }
\def \J {\mathbf{1}\mathbf{1}^T}

\newcommand{\op}[1]{\operatorname{#1}}
\newcommand{\setdef}[2]{ := \keys{ #1 \; | \; #2 } }
\newcommand{\set}[2]{ \keys{ #1 \; | \; #2 } }
\newcommand{\sign}[1]{\op{sign}\left( #1 \right) }
\newcommand{\trace}[1]{\op{tr}\left( #1 \right) }
\newcommand{\tr}[1]{\op{tr}\left( #1 \right) }
\newcommand{\inv}[1]{\left( #1 \right)^{-1} }
\newcommand{\abs}[1]{\left| #1 \right|}
\newcommand{\sabs}[1]{| #1 |}
\newcommand{\keys}[1]{\left\{ #1 \right\}}
\newcommand{\sqbr}[1]{\left[ #1 \right]}
\newcommand{\ssqbr}[1]{ [ #1  ]}
\newcommand{\sbrac}[1]{ ( #1 ) }
\newcommand{\brac}[1]{\left( #1 \right) }
\newcommand{\bbrac}[1]{\big( #1 \big) }
\newcommand{\Bbrac}[1]{\Big( #1 \Big)}
\newcommand{\BBbrac}[1]{\BIG( #1 \Big)}
\newcommand{\MAT}[1]{\begin{bmatrix} #1 \end{bmatrix}}
\newcommand{\sMAT}[1]{\left(\begin{smallmatrix} #1 \end{smallmatrix}\right)}
\newcommand{\sMATn}[1]{\begin{smallmatrix} #1 \end{smallmatrix}}
\newcommand{\PROD}[2]{\left \langle #1, #2\right \rangle}
\newcommand{\PRODs}[2]{\langle #1, #2 \rangle}
\newcommand{\der}[2]{\frac{\text{d}#2}{\text{d}#1}}
\newcommand{\pder}[2]{\frac{\partial#2}{\partial#1}}
\newcommand{\derTwo}[2]{\frac{\text{d}^2#2}{\text{d}#1^2}}
\newcommand{\ceil}[1]{\lceil #1 \rceil}
\newcommand{\Imag}[1]{\op{Im}\brac{ #1 }}
\newcommand{\Real}[1]{\op{Re}\brac{ #1 }}
\newcommand{\norm}[1]{\left|\left| #1 \right|\right| }
\newcommand{\norms}[1]{ \| #1 \|  }
\newcommand{\normProd}[1]{\left|\left| #1 \right|\right| _{\PROD{\cdot}{\cdot}} }
\newcommand{\normTwo}[1]{\left|\left| #1 \right|\right| _{2} }
\newcommand{\normTwos}[1]{ \| #1  \| _{2} }
\newcommand{\normZero}[1]{\left|\left| #1 \right|\right| _{0} }
\newcommand{\normTV}[1]{\left|\left| #1 \right|\right|  _{ \op{TV}  } }% _{\op{c} \ell_1} }
\newcommand{\normOne}[1]{\left|\left| #1 \right|\right| _{1} }
\newcommand{\normOnes}[1]{\| #1 \| _{1} }
\newcommand{\normOneTwo}[1]{\left|\left| #1 \right|\right| _{1,2} }
\newcommand{\normF}[1]{\left|\left| #1 \right|\right| _{\op{F}} }
\newcommand{\normLTwo}[1]{\left|\left| #1 \right|\right| _{\ml{L}_2} }
\newcommand{\normNuc}[1]{\left|\left| #1 \right|\right| _{\ast} }
\newcommand{\normOp}[1]{\left|\left| #1 \right|\right|  }
\newcommand{\normInf}[1]{\left|\left| #1 \right|\right| _{\infty}  }
\newcommand{\proj}[1]{\mathcal{P}_{#1} \, }
\newcommand{\diff}[1]{ \, \text{d}#1 }
\newcommand{\vc}[1]{\boldsymbol{\vec{#1}}}
\newcommand{\rc}[1]{\boldsymbol{#1}}
\newcommand{\vx}{\vec{x}}
\newcommand{\vy}{\vec{y}}
\newcommand{\vz}{\vec{z}}
\newcommand{\vu}{\vec{u}}
\newcommand{\vv}{\vec{v}}
\newcommand{\vb}{\vec{\beta}}
\newcommand{\va}{\vec{\alpha}}
\newcommand{\vaa}{\vec{a}}
\newcommand{\vbb}{\vec{b}}
\newcommand{\vg}{\vec{g}}
\newcommand{\vw}{\vec{w}}
\newcommand{\vh}{\vec{h}}
\newcommand{\vbeta}{\vec{\beta}}
\newcommand{\valpha}{\vec{\alpha}}
\newcommand{\vgamma}{\vec{\gamma}}
\newcommand{\veta}{\vec{\eta}}
\newcommand{\vnu}{\vec{\nu}}
\newcommand{\rw}{\rnd{w}}
\newcommand{\rvnu}{\vc{\nu}}
\newcommand{\rvv}{\rndvec{v}}
\newcommand{\rvw}{\rndvec{w}}
\newcommand{\rvx}{\rndvec{x}}
\newcommand{\rvy}{\rndvec{y}}
\newcommand{\rvz}{\rndvec{z}}
\newcommand{\rvX}{\rndvec{X}}


\newtheorem{theorem}{Theorem}[section]
% \declaretheorem[style=plain,qed=$\square$]{theorem}
\newtheorem{corollary}[theorem]{Corollary}
\newtheorem{definition}[theorem]{Definition}
\newtheorem{lemma}[theorem]{Lemma}
\newtheorem{remark}[theorem]{Remark}
\newtheorem{algorithm}[theorem]{Algorithm}

% \theoremstyle{definition}
%\newtheorem{example}[proof]{Example}
\declaretheorem[style=definition,qed=$\triangle$,sibling=definition]{example}
\declaretheorem[style=definition,qed=$\bigcirc$,sibling=definition]{application}

%
%% Typographic tweaks and miscellaneous
%\newcommand{\sfrac}[2]{\mbox{\small$\displaystyle\frac{#1}{#2}$}}
%\newcommand{\suchthat}{\kern0.1em{:}\kern0.3em}
%\newcommand{\qqquad}{\kern3em}
%\newcommand{\cond}{\,|\,}
%\def\Matlab{\textsc{Matlab}}
%\newcommand{\displayskip}[1]{\abovedisplayskip #1\belowdisplayskip #1}
%\newcommand{\term}[1]{\emph{#1}}
%\renewcommand{\implies}{\;\Rightarrow\;}



\newcommand{\ru}{\rnd{ u}  }
\newcommand{\rd}{\rnd{ d}  }
%\newcommand{\rs}{\rnd{ s}  }
\newcommand{\ri}{\rnd{ i}  }
\newcommand{\re}{\rnd{ e}  }
\newcommand{\rQ}{\rnd{ q}  }
\newcommand{\rC}{\rnd{ c}  }


\begin{document}

\begin{center}
{\large{\textbf{Homework 4}} } \vspace{0.2cm}\\
Due October 15 at 11 pm
\\
\end{center}
Unless stated otherwise, justify any answers you give.
You can work in groups, but each
student must write their own solution based on their own
understanding of the problem.

When uploading your homework to Gradescope you will have to
select the relevant pages for each question.  Please submit each
problem on a separate page (i.e., 1a and~1b can be on the same page but 1
and 2 must be on different pages).  We understand that this may be
cumbersome but this is the best way for the grading team to grade your
homework assignments and provide feedback in a timely manner.  Failure
to adhere to these guidelines may result in a loss of points.
Note that it may take some time to
select the pages for your submission.  Please plan accordingly.  We
suggest uploading your assignment at least 30 minutes before the deadline
so you will have ample time to select the correct pages for your
submission.  If you are using \LaTeX, consider using the minted or
listings packages for typesetting code.  
\\

\begin{enumerate}

\item (Half life)
The half life of a radioactive substance is a way to quantify how rapidly the substance decays. Given a fixed quantity of the substance, the half time is the time that it takes for it to be reduced to half (i.e. half of the radioactive particles have decayed). It is not immediately apparent why the time should be the same for any quantity. Here we show that it is (probabilistically), as long the particles decay following an exponential distribution.  
\begin{enumerate}
\item Let $\rnd{t}$ be a random variable with a pdf of the form
\begin{align}
f_{\rnd{t}}(t) := \begin{cases}
\lambda \exp(- \lambda t), \qquad \text{if $t\geq 0$},\\
0 \qquad \text{otherwise},
\end{cases}
\end{align}
where $\lambda$ is a fixed constant. We define the half life $t_{1/2}$ as the number that satisfies $\P(\rnd{t} > t_{1/2}) = 1/2$. Compute $t_{1/2}$ in terms of $\lambda$. Then explain intuitively why this is a reasonable definition for the half life.
\item Compute $t$ such that $\P( t_{1/2} < \rnd{t} < t) = 1/4$, and express it in terms of only $t_{1/2}$. Explain why the result is consistent with the intuitive meaning of half life.
\item Compute $\P( \rnd{t} > k t_{1/2} )$ for any integer $k$. Again, explain why the result is consistent with the intuitive meaning of half life.
\end{enumerate}

\item (Triangular pdf)
We are interested in fitting a model with a parametric pdf equal to
\begin{align}
f_{w}\brac{x}  = \begin{cases}
 \frac{2x}{w^2}, \qquad & \text{for } 0 \leq x \leq w,\\
0, \qquad & \text{otherwise},
\end{cases}
\end{align}
where the parameter $w$ is nonnegative. 
%\begin{comment}
%Both the pdf and the cdf are plotted in Figure~\ref{fig:triangle}.  
%\begin{figure}[tp]
%% Preamble: \pgfplotsset{width=7cm,compat=1.12}
%% \begin{center}
%\begin{tikzpicture}[scale=0.95]
%\begin{axis}[xmin= -0.1, xmax=2.1, ymin=-0.1, ymax=1.1,xlabel=$x$,
%ylabel=$f_{X} \brac{x }  $, xticklabels={0,w}, xtick={0,2},
%yticklabels={0,$\frac{2}{w}$}, ytick={0,1}]
%\addplot[blue, very thick, domain=0:2, samples=51] { x/2)};
%\addplot[blue, very thick, domain=-0.1:0, samples=2] {0};
%\addplot[blue, very thick, domain=2:2.1, samples=2] {0};
%\addplot[dashed, very thick, samples=2] coordinates {(2,0)(2,1)};
%\end{axis}
%\end{tikzpicture}
%%\end{center}
%\caption{Triangular pdf and the corresponding cdf.}
%\label{fig:triangle}
%\end{figure}
%\end{comment}

\begin{enumerate}
\item The observed values are 1.25, 0.4, 1.5, 1, 1.2. What are the possible values of the parameter $w$? 
\item Compute the likelihood function corresponding to these data and sketch it. 
\item What is the maximum likelihood estimate of $w$? 
\item  Assume that the data are indeed generated by the parametric model with $w:=w_{\op{true}}$. Does the ML estimate systematically underestimate or overestimate the true parameter? 
\item Generate a sample from a random variable with this parametric distribution, where $w:=2$, using a uniform sample from the interval $\sqbr{0,1}$ equal to 0.64.
\end{enumerate}

\item (Planet)
An astrophysicist determines that a good model for the pdf of the temperature in a newly discovered planet is
\begin{align}
f_{\rnd{t}}(t) := \frac{\lambda \exp (-\lambda \abs{t})}{2},
\end{align}
where $t$ can be any real number (in particular it can be negative or positive).
\begin{enumerate}
\item Compute the cdf of $\rnd{t}$.
\item Compute the maximum-likelihood estimate of $\lambda$ from the following data: 5, -50, -1, 100
\item What is the pdf of $\rnd{t}$ conditioned on the event $\rnd{t}>0$?
\end{enumerate}

\item(Temperature)
The tables in \textit{train.csv} and \textit{test.csv} record the daily maximum temperature (TMAX) of Seattle.
\begin{enumerate}
    \item Estimate the pdf of TMAX with the following models on the training set. Compare the pdf with a normalized histogram in the test set. Which model performs better visually?
        \begin{itemize} 
            \item Estimating the parameter of Gaussian distribution with MLE;
            \item Non-parametric KDE with the Gaussian kernel at different bandwidths (e.g. 1, 2, 5).
        \end{itemize}
    \item Repeat the experiment only on \text{July} and \text{August} data. Which model performs better visually? Compare the results with (a) and explain your findings. 
\end{enumerate}

\end{enumerate}
\end{document}
