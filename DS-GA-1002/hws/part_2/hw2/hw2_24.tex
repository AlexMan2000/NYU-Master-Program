\documentclass[12pt,twoside]{article}
\usepackage[dvipsnames]{xcolor}
\usepackage{tikz,graphicx,amsmath,amsfonts,amscd,amssymb,bm,cite,epsfig,epsf,url}
\usepackage[hang,flushmargin]{footmisc}
\usepackage[colorlinks=true,urlcolor=blue,citecolor=blue]{hyperref}
\usepackage{amsthm,multirow,wasysym,appendix}
\usepackage{array,subcaption} 
% \usepackage[small,bf]{caption}
\usepackage{bbm}
\usepackage{pgfplots}
\usetikzlibrary{spy}
\usepgfplotslibrary{external}
\usepgfplotslibrary{fillbetween}
\usetikzlibrary{arrows,automata}
\usepackage{thmtools}
\usepackage{blkarray} 
\usepackage{textcomp}
\usepackage[left=0.8in,right=1.0in,top=1.0in,bottom=1.0in]{geometry}

%% Probability operators and functions
%
% \def \P{\mathrm{P}}
\def \P{\mathrm{P}}
\def \E{\mathrm{E}}
\def \Var{\mathrm{Var}}
\let\var\Var
\def \Cov {\mathrm{Cov}} \let\cov\Cov
\def \MSE {\mathrm{MSE}} \let\mse\MSE
\def \sgn {\mathrm{sgn}}
\def \R {\mathbb{R}}
\def \C {\mathbb{C}}
\def \N {\mathbb{N}}
\def \Z {\mathbb{Z}}
\def \cV {\mathcal{V}}
\def \cS {\mathcal{S}}

\newcommand{\RR}{\ensuremath{\mathbb{R}}}

\DeclareMathOperator*{\argmin}{arg\,min}
\DeclareMathOperator*{\argmax}{arg\,max}
\newcommand{\red}[1]{\textcolor{red}{#1}}
\newcommand{\blue}[1]{\textcolor{blue}{#1}}
\newcommand{\green}[1]{\textcolor{ForestGreen}{ #1}}
\newcommand{\fuchsia}[1]{\textcolor{RoyalPurple}{ #1}}

\newcommand{\wrnd}[1]{\widetilde{ #1 } }
\newcommand{\po}{\wrnd{\op{po}}  }

%
%% Probability distributions
%
%\def \Bern    {\mathrm{Bern}}
%\def \Binom   {\mathrm{Binom}}
%\def \Exp     {\mathrm{Exp}}
%\def \Geom    {\mathrm{Geom}}
% \def \Norm    {\mathcal{N}}
%\def \Poisson {\mathrm{Poisson}}
%\def \Unif    {\mathrm {U}}
%
\DeclareMathOperator{\Norm}{\mathcal{N}}

\newcommand{\bdb}[1]{\textcolor{red}{#1}}

\newcommand{\ml}[1]{\mathcal{ #1 } }
\newcommand{\wh}[1]{\widehat{ #1 } }
\newcommand{\wt}[1]{\widetilde{ #1 } }
\newcommand{\conj}[1]{\overline{ #1 } }
\newcommand{\rnd}[1]{\tilde{ #1 } }
\newcommand{\rv}[1]{ \rnd{ #1}  }
\newcommand{\rM}{\rnd{ m}  }
\newcommand{\rx}{\rnd{ x}  }
\newcommand{\ry}{\rnd{ y}  }
\newcommand{\rz}{\rnd{ z}  }
\newcommand{\ra}{\rnd{ a}  }
\newcommand{\rb}{\rnd{ b}  }
\newcommand{\rt}{\rnd{ t}  }
\newcommand{\rs}{\rnd{ s}  }


\newcommand{\rpc}{\widetilde{ pc}  }
\newcommand{\rndvec}[1]{\vec{\rnd{#1}}}

\def \cnd {\, | \,}
\def \Id { I }
\def \J {\mathbf{1}\mathbf{1}^T}

\newcommand{\op}[1]{\operatorname{#1}}
\newcommand{\setdef}[2]{ := \keys{ #1 \; | \; #2 } }
\newcommand{\set}[2]{ \keys{ #1 \; | \; #2 } }
\newcommand{\sign}[1]{\op{sign}\left( #1 \right) }
\newcommand{\trace}[1]{\op{tr}\left( #1 \right) }
\newcommand{\tr}[1]{\op{tr}\left( #1 \right) }
\newcommand{\inv}[1]{\left( #1 \right)^{-1} }
\newcommand{\abs}[1]{\left| #1 \right|}
\newcommand{\sabs}[1]{| #1 |}
\newcommand{\keys}[1]{\left\{ #1 \right\}}
\newcommand{\sqbr}[1]{\left[ #1 \right]}
\newcommand{\ssqbr}[1]{ [ #1  ]}
\newcommand{\sbrac}[1]{ ( #1 ) }
\newcommand{\brac}[1]{\left( #1 \right) }
\newcommand{\bbrac}[1]{\big( #1 \big) }
\newcommand{\Bbrac}[1]{\Big( #1 \Big)}
\newcommand{\BBbrac}[1]{\BIG( #1 \Big)}
\newcommand{\MAT}[1]{\begin{bmatrix} #1 \end{bmatrix}}
\newcommand{\sMAT}[1]{\left(\begin{smallmatrix} #1 \end{smallmatrix}\right)}
\newcommand{\sMATn}[1]{\begin{smallmatrix} #1 \end{smallmatrix}}
\newcommand{\PROD}[2]{\left \langle #1, #2\right \rangle}
\newcommand{\PRODs}[2]{\langle #1, #2 \rangle}
\newcommand{\der}[2]{\frac{\text{d}#2}{\text{d}#1}}
\newcommand{\pder}[2]{\frac{\partial#2}{\partial#1}}
\newcommand{\derTwo}[2]{\frac{\text{d}^2#2}{\text{d}#1^2}}
\newcommand{\ceil}[1]{\lceil #1 \rceil}
\newcommand{\Imag}[1]{\op{Im}\brac{ #1 }}
\newcommand{\Real}[1]{\op{Re}\brac{ #1 }}
\newcommand{\norm}[1]{\left|\left| #1 \right|\right| }
\newcommand{\norms}[1]{ \| #1 \|  }
\newcommand{\normProd}[1]{\left|\left| #1 \right|\right| _{\PROD{\cdot}{\cdot}} }
\newcommand{\normTwo}[1]{\left|\left| #1 \right|\right| _{2} }
\newcommand{\normTwos}[1]{ \| #1  \| _{2} }
\newcommand{\normZero}[1]{\left|\left| #1 \right|\right| _{0} }
\newcommand{\normTV}[1]{\left|\left| #1 \right|\right|  _{ \op{TV}  } }% _{\op{c} \ell_1} }
\newcommand{\normOne}[1]{\left|\left| #1 \right|\right| _{1} }
\newcommand{\normOnes}[1]{\| #1 \| _{1} }
\newcommand{\normOneTwo}[1]{\left|\left| #1 \right|\right| _{1,2} }
\newcommand{\normF}[1]{\left|\left| #1 \right|\right| _{\op{F}} }
\newcommand{\normLTwo}[1]{\left|\left| #1 \right|\right| _{\ml{L}_2} }
\newcommand{\normNuc}[1]{\left|\left| #1 \right|\right| _{\ast} }
\newcommand{\normOp}[1]{\left|\left| #1 \right|\right|  }
\newcommand{\normInf}[1]{\left|\left| #1 \right|\right| _{\infty}  }
\newcommand{\proj}[1]{\mathcal{P}_{#1} \, }
\newcommand{\diff}[1]{ \, \text{d}#1 }
\newcommand{\vc}[1]{\boldsymbol{\vec{#1}}}
\newcommand{\rc}[1]{\boldsymbol{#1}}
\newcommand{\vx}{\vec{x}}
\newcommand{\vy}{\vec{y}}
\newcommand{\vz}{\vec{z}}
\newcommand{\vu}{\vec{u}}
\newcommand{\vv}{\vec{v}}
\newcommand{\vb}{\vec{\beta}}
\newcommand{\va}{\vec{\alpha}}
\newcommand{\vaa}{\vec{a}}
\newcommand{\vbb}{\vec{b}}
\newcommand{\vg}{\vec{g}}
\newcommand{\vw}{\vec{w}}
\newcommand{\vh}{\vec{h}}
\newcommand{\vbeta}{\vec{\beta}}
\newcommand{\valpha}{\vec{\alpha}}
\newcommand{\vgamma}{\vec{\gamma}}
\newcommand{\veta}{\vec{\eta}}
\newcommand{\vnu}{\vec{\nu}}
\newcommand{\rw}{\rnd{w}}
\newcommand{\rvnu}{\vc{\nu}}
\newcommand{\rvv}{\rndvec{v}}
\newcommand{\rvw}{\rndvec{w}}
\newcommand{\rvx}{\rndvec{x}}
\newcommand{\rvy}{\rndvec{y}}
\newcommand{\rvz}{\rndvec{z}}
\newcommand{\rvX}{\rndvec{X}}


\newtheorem{theorem}{Theorem}[section]
% \declaretheorem[style=plain,qed=$\square$]{theorem}
\newtheorem{corollary}[theorem]{Corollary}
\newtheorem{definition}[theorem]{Definition}
\newtheorem{lemma}[theorem]{Lemma}
\newtheorem{remark}[theorem]{Remark}
\newtheorem{algorithm}[theorem]{Algorithm}

% \theoremstyle{definition}
%\newtheorem{example}[proof]{Example}
\declaretheorem[style=definition,qed=$\triangle$,sibling=definition]{example}
\declaretheorem[style=definition,qed=$\bigcirc$,sibling=definition]{application}

%
%% Typographic tweaks and miscellaneous
%\newcommand{\sfrac}[2]{\mbox{\small$\displaystyle\frac{#1}{#2}$}}
%\newcommand{\suchthat}{\kern0.1em{:}\kern0.3em}
%\newcommand{\qqquad}{\kern3em}
%\newcommand{\cond}{\,|\,}
%\def\Matlab{\textsc{Matlab}}
%\newcommand{\displayskip}[1]{\abovedisplayskip #1\belowdisplayskip #1}
%\newcommand{\term}[1]{\emph{#1}}
%\renewcommand{\implies}{\;\Rightarrow\;}



\begin{document}

\begin{center}
{\large{\textbf{Homework 2}} } \vspace{0.2cm}\\
Due Feb 11 at 11 pm
\\
\end{center}
Unless stated otherwise, justify any answers you give.
You can work in groups, but each
student must write their own solution based on their own
understanding of the problem.

When uploading your homework to Gradescope you will have to
select the relevant pages for each question.  Please submit each
problem on a separate page (i.e., 1a and~1b can be on the same page but 1
and 2 must be on different pages).  We understand that this may be
cumbersome but this is the best way for the grading team to grade your
homework assignments and provide feedback in a timely manner.  Failure
to adhere to these guidelines may result in a loss of points.
Note that it may take some time to
select the pages for your submission.  Please plan accordingly.  We
suggest uploading your assignment at least 30 minutes before the deadline
so you will have ample time to select the correct pages for your
submission.  If you are using \LaTeX, consider using the minted or
listings packages for typesetting code.  
\\

\begin{enumerate}

\item (Properties of covariance) 
Prove the following properties of covariance (you can use properties established in the chapter).\\ 

For any random variables $\ra$ and $\rb$ with finite variance:
\begin{enumerate} 
\item For any $\alpha$, $\beta \in \R$
\begin{align}
\cov \ssqbr{\beta \ra + \alpha,\rb} & = \beta \cov \ssqbr{\ra,\rb}. 
\end{align} 
\item 
\begin{align}
\cov \ssqbr{\ra+\rb,\ra -\rb} = \var \ssqbr{\ra} - \var \ssqbr{\rb}.
\end{align}
\end{enumerate}

\item (Standardized variables and the sample correlation coefficient) 
We study the sample correlation coefficient $\rho_{X,Y} $ from Definition~8.10. 

We denote the OLS estimator of $y_i$, given $x_i$ by $\ell_{\op{OLS}} (x_i)$ and the corresponding residual by 
\begin{align}
r_i := y_i -  \ell_{\op{OLS}} (x_i), \qquad 1\leq i \leq n.
\end{align}
(\emph{Hint: For all the proofs, follow the same arguments as in the chapter, replacing the expectation operator by the averaging operator.})
\begin{enumerate} 
\item For the standardized data
\begin{align}
s(x_i) & := \frac{x_i - m(X)}{\sqrt{v(X)}}, \\
s(y_i) & := \frac{y_i - m(Y)}{\sqrt{v(Y)}}, \quad 1 \leq i \leq n,
\end{align}
where $m(X)$ and $m( Y)$ are the sample means of $X$ and $Y$, and $v(X)$ and $v(Y)$ the sample variances, we define the standardized datasets as $S_X:= \keys{s(x_1),s(x_2),\ldots,s(x_n)}$ and $S_Y:= \keys{s(y_1),s(y_2),\ldots,s(y_n)}$. Show that the sample mean of the standardized data is zero,
\begin{align}
m(S_X) = m(S_Y)=0,
\end{align}
the sample variance is one,
\begin{align}
v(S_X) & = \frac{1}{n-1}\sum_{i=1}^{n}s(x_i)^2 = 1,\\
v(S_Y) & = \frac{1}{n-1}\sum_{i=1}^{n}s(y_i)^2 =1,
\end{align}
and the sample covariance is equal to the sample correlation coefficient of the original data,
\begin{align}
c(S_X,S_Y)=\frac{1}{n-1}\sum_{i=1}^{n}s(x_i)s(y_i) = \rho_{X,Y}.
\end{align}
\item Prove that 
\begin{align}
\frac{1}{n-1} \sum_{i=1}^{n} r_i^2 & = \brac{1 - \rho_{X,Y}^2} v(Y).
\end{align}
\item Prove that the sample correlation coefficient satisfies the same bounds as the correlation coefficient,
\begin{align}
-1 \leq \rho_{X,Y} \leq 1,
\end{align}
as long as $v(Y)$ is not zero.
\item Prove that if $\rho_{X,Y} = \pm 1$, then $y_i = \beta x_i + \alpha$, $1 \leq i \leq n$, for some constant $\alpha$, $\beta \in \R$.
\end{enumerate}

\item (Height and Weight)
The table in \texttt{ANSUR II MALE Public.csv} reports physical measurements of members of the US army. In this problem, we will work with simple linear regression to estimate weight (\textit{Weightlbs}) as a function of height (\textit{Heightin}). Let \(h\) represent height (in inches) and \(w\) represent weight (in pounds).

\begin{enumerate}
	\item Compute the OLS estimator of weight given height. Add this line on a scatterplot (with $h$ on the x-axis; $w$ on the y-axis) and also provide the OLS estimator in equation form:
\[
\hat{w} = \hat{\beta}_0 + \hat{\beta}_1 h
\]


	\item Compute the sample covariance between the height and the residual of the OLS estimator. Are they correlated?
    \item Interpret the slope coefficient, $\hat{\beta}_1$.	
	\item Compute the sample variance of the weight, the OLS estimator of the weight, and the residual. What relationship do you find among these three values?
	\item Compute and compare the sample coefficient of determination (using its definition as the fraction of the variance explained by the linear estimator), and compare it to the squared sample correlation coefficient.
	\item Compute the OLS estimator of \(h\) given \(w\). Then rearrange this formula to express \(w\) as a function of \(h\).
\[
\tilde{w} = \tilde{\beta}_0 + \tilde{\beta}_1 h
\]	
	 Add this line also on the scatterplot (in a different color), and compare it with your initial OLS line.
\end{enumerate}


\end{enumerate}
\end{document}
